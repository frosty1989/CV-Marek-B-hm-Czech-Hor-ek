%%%%%%%%%%%%%%%%%
% This is an sample CV template created using altacv.cls
% (v1.1.5, 1 December 2018) written by LianTze Lim (liantze@gmail.com). Now compiles with pdfLaTeX, XeLaTeX and LuaLaTeX.
%
%% It may be distributed and/or modified under the
%% conditions of the LaTeX Project Public License, either version 1.3
%% of this license or (at your option) any later version.
%% The latest version of this license is in
%%    http://www.latex-project.org/lppl.txt
%% and version 1.3 or later is part of all distributions of LaTeX
%% version 2003/12/01 or later.
%%%%%%%%%%%%%%%%


%% If you need to pass whatever options to xcolor
\PassOptionsToPackage{dvipsnames}{xcolor}

%% If you are using \orcid or academicons
%% icons, make sure you have the academicons
%% option here, and compile with XeLaTeX
%% or LuaLaTeX.
% \documentclass[10pt,a4paper,academicons]{altacv}

%% Use the "normalphoto" option if you want a normal photo instead of cropped to a circle
% \documentclass[10pt,a4paper,normalphoto]{altacv}

\documentclass[10pt,a4paper,ragged2e]{altacv}

%% AltaCV uses the fontawesome and academicon fonts
%% and packages.
%% See texdoc.net/pkg/fontawecome and http://texdoc.net/pkg/academicons for full list of symbols. You MUST compile with XeLaTeX or LuaLaTeX if you want to use academicons.

% Change the page layout if you need to
\geometry{left=1cm,right=9cm,marginparwidth=6.8cm,marginparsep=1.2cm,top=1.25cm,bottom=1.25cm}

% Change the font if you want to, depending on whether
% you're using pdflatex or xelatex/lualatex
\ifxetexorluatex
  % If using xelatex or lualatex:
  \setmainfont{Carlito}
\else
  % If using pdflatex:
  \usepackage[utf8]{inputenc}
  \usepackage[T1]{fontenc}
  \usepackage[default]{lato}
\fi

% Change the colours if you want to
\definecolor{Mulberry}{HTML}{72243D}
\definecolor{SlateGrey}{HTML}{2E2E2E}
\definecolor{LightGrey}{HTML}{666666}
\colorlet{heading}{Sepia}
\colorlet{accent}{Mulberry}
\colorlet{emphasis}{SlateGrey}
\colorlet{body}{LightGrey}

% Change the bullets for itemize and rating marker
% for \cvskill if you want to
\renewcommand{\itemmarker}{{\small\textbullet}}
\renewcommand{\ratingmarker}{\faCircle}

%% sample.bib contains your publications
\addbibresource{sample.bib}

\begin{document}
\name{Marek Böhm}
\tagline{PhD student}
\photo{2.8cm}{marekbohm}
\personalinfo{%
  % Not all of these are required!
  % You can add your own with \printinfo{symbol}{detail}
  \email{marek.bhm@seznam.cz}
  \phone{+420 606 259 414}
  \mailaddress{Libušská 408/100a, Praha-Písnice, 142 00}

  %% You MUST add the academicons option to \documentclass, then compile with LuaLaTeX or XeLaTeX, if you want to use \orcid or other academicons commands.
  % \orcid{orcid.org/0000-0000-0000-0000}
}

%% Make the header extend all the way to the right, if you want.
\begin{fullwidth}
\makecvheader
\end{fullwidth}

%% Depending on your tastes, you may want to make fonts of itemize environments slightly smaller
% \AtBeginEnvironment{itemize}{\small}

%% Provide the file name containing the sidebar contents as an optional parameter to \cvsection.
%% You can always just use \marginpar{...} if you do
%% not need to align the top of the contents to any
%% \cvsection title in the "main" bar.
\cvsection[page1sidebar]{Praxe}

\cvevent{PhD student}{centrum HiLASE}{Červen 2014 -- dosud}{Dolní Břežany, Česká republika}
\begin{itemize}

\item Vývoj řídicího systému pro Laser Shock Peening stanici v centru HiLASE, Fyzikální ústav Akademie věd České republiky

\end{itemize}

\cvsection{Projekty}





\cvevent{DOLASTOOL - Vývoj a optimalizace laserových aditivních, subtraktivních a transformačních technologií pro nástrojářský průmysl}{Technologická agentura České republiky}{leden 2020 -- prosinec 2022}{}{}
Cílem projektu je zpracování tvrdých nástrojových materiálů pomocí laseru za účelem zlepšení jejich užitných vlastností nebo životnosti.

\cvsection{Zkušenosti s programováním}

\cvevent{Python}{}{}{}
\begin{itemize}
\item Tvorba skriptů pro postprocesor průmyslového robotického ramene
\end{itemize}

\divider

\cvevent{Programování průmyslových robotických ramen FANUC}{}{}{}
\begin{itemize}
\item Programování průmyslových robotických ramen FANUC pomocí CAM systému Roboguide 
\end{itemize}

\divider

\cvevent{LabVIEW}{}{}{}
\begin{itemize}
\item LabVIEW VI pro lineární pohony
\item LabVIEW VI pro robotické rameno FANUC
\end{itemize}

\divider

\cvevent{PLC Schneider Zelio Logic}{}{}{}
\begin{itemize}
\item Práce s prostředím Zelio Soft 2

\end{itemize}



\medskip



\cvsection{Publikace}

\nocite{*}

\printbibliography[heading=pubtype,title={\printinfo{\faFileTextO}{Články v odborných časopisech}},type=article]

\cvsection{Technické znalosti a dovednosti}


\cvtag{Elektronika}
\cvtag{Optika}
\cvtag{Laserová fyzika}

\divider\smallskip

\cvtag{LabVIEW}
\cvtag{Python}
\cvtag{Programování průmyslových robotických ramen}
\cvtag{CAM systémy}
\cvtag{Programování PLC}
\cvtag{Javascript}
\cvtag{HTML}
\cvtag{CSS}




%% If the NEXT page doesn't start with a \cvsection but you'd
%% still like to add a sidebar, then use this command on THIS
%% page to add it. The optional argument lets you pull up the
%% sidebar a bit so that it looks aligned with the top of the
%% main column.
% \addnextpagesidebar[-1ex]{page3sidebar}


\end{document}
